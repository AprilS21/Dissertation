\chapter{Introduction}

Introduction to the material covered in the document.

\section{Background for GAEN keys}
\label{sec:Background}

Background
Here is summaries of the papers I have read thus far

The first paper is Contact Tracing App Privacy: What Data Is Shared
By Europe’s GAEN Contact Tracing Apps~(\cite{9488728}). This paper discusses the data sent to back end servers of various contact tracing apps in Europe. The apps have two components: the 'client' app which is controlled by the national public health authority and the GAEN service that, on android, is managed by Google as part of the Google Play services. The paper investigates both of these components. It found that the Google Play services component continuously sends requests to the Google servers which contain information such as the phone IMEI, handset hardware serial number, SIM serial number, etc. This may be considered intrusive , however most users would have accepted this data sharing if that had previously enabled Google Play services. This data sharing is not specific to the covid tracing app. There may be a way to identify a user is  specific data is sent in every request. There are experiments run to test the data being sent by different apps. 

The paper CoAvoid: Secure, Privacy-Preserved Tracing of
Contacts for Infectious Diseases~(\cite{9908579}) discusses the privacy and security concerns of covid tracing apps and proposes its own CoAvoid app which improves on these issues. It talks about how inappropriate uses of the GAEN API may expose all information about confirmed patients to servers and relevant users. This may allow someone to gather lots of information about patients and discover their identities, daily routines or social relationships. Due to a limitation of BLE, it may be possible for attackers to send users excess false alarms and put further strain on the public health system (wormhole attack). It talk about how the BLE to calculate if two devices came into contact can be affected by factors such as environment, transmitting power, receiving sensitivity, etc. The GAEN design is vulnerable to profiling, possibly de-anonymising infected persons and wormhole attacks. It describes how the GAEN app works: It randomly generates a Daily Tracking Key (DTK), a unique identifier used for 24 hours. It uses function {\it f} to derive the Rolling Proximity Identifier (RPI) based on the DTK. These identifiers are sent in Bluetooth Advertisements, which will be replaced every 20 minutes. The apps simultaneously collects and stores other users RPIs locally. If a user test positive, their DTK is uploaded to a central server. Other download these DTKs and reconstruct the RPIs. They compare these to their local list of keys. The authenticity of the information being downloaded cannot be verified and thus wormhole attacks may occur. The paper further describes two types of potential attacks: Wormhole and Privacy Analysis attack. It may be possible to identify things about a user by tracking the DTK uploads.

The paper Digital Contact Tracing Solutions: Promises, Pitfalls
and Challenges~(\cite{9931613}) analyses digital contact tracing solutions. It discusses the security and privacy issues with GAEN. It proposes its own solution called TraceCorona. Apple and Google collaborated to create a decentralised contact tracing interface calle Exposure Notification API (GAEN). Access to the API has been given to only one organisation authorised by the government. BLE is used for sensing the proximity between individual devices. The phones send out information like temporary identifiers (TempIDs) that can be sensed by other devices. It also records signal strength in an attempt to estimate the distance of the encounter. It discusses requirements of accuracy, superspreader and accountability for a digital contact tracing system to be acceptable.

The article Privacy and Integrity Threats in Contact
Tracing Systems and Their Mitigations~(\cite{9928557}) reports privacay and integrity threats in GAEN and proposes a new system called Pronto-B2. Threats to security and privacy include the possibility of tracing and deanonymising citizens using passive devices and injecting false at-risk alerts. An attacker may trace a user by linking locations visited by the same user. They may also try to deanonymise users by linking locations visited by users to their real identities. 'Paparazzi Attack': using passive devices, the attacker can catch and store the pseudonyms of a target user. They can link together the passive devices recieved the pseudonyms belonging to the same user. The attacker can obtain information about the habits of an infected user and use it for economical gain. 'Brutus Attack': The attacker colludes with the server and the health authority to discover which user uploaded certain data. Pseudonyms in GAEN are called rolling proximity identifiers (RPIs). A short random secret called Temporary Exposure key (TEK) is generated each day by the smartphone. All RPIs of a given day of a user are generated by running a PRF on the input TEK.

The paper October 2020 Survey of GAEN App Key Uploads~(\cite{survey}) is a survey of the TEKs published across 8 European regions. It estimates the number of users uploading TEKs and compares this to the expected number based on population, number of active users and covid case counts. It reports a shortfall of uploads in a number of regions. The efficacy of these apps remains unclear.

\section{Style of English}
\label{sec:StyleOfEnglish}

Style of English
An impersonal style keeps the technical factors and ideas to the forefront of the discussion and you in the background. Try to be objective and quantitative in your conclusions. For example, it is not enough to say vaguely “because the compiler was unreliable the code produced was not adequate”. It would be much better to say “because the XYZ compiler produced code which ran 2-3 times slower than PQR (see Table x,y), a fast enough scheduler could not be written using this algorithm”. The second version is more likely to make the reader think the writer knows what he/she is talking about, since it is a lot more authoritative. Also, you will not be able to write the second version without a modicum of thought and effort.

The following points are couple of {\it Do's \& Dont's} that I have noted down as feedback to reports over the years. The focus of this list is to encourage writers to be specific in writing reports - some of this is motivated by Strunk and White's The Elements of Style~(\cite{9908579}). Regarding reports that are submitted as part of a degree, examiners have to read and mark these reports - make it easy for these examiners to give good marks by following a number of simple points:

\begin{description}
	\item [Acronyms:] Acronyms should be introduced by the words they represent followed by the acronym in capitals enclosed in brackets e.g. "...TCP (Transmission Control Protocol)..." $\Rightarrow$  "... Transmission Control Protocol (TCP)..."
	\item [Contractions:] I would generally suggest to avoid contractions such as "I'd", "They've", etc in reports. In some cases, they are ambiguous e.g. "I'd" $\Rightarrow$ "I would" or "I had" and can lead to misunderstandings.
	\item [Avoid "do":] Be specific and use specific verbs to describe actions.
	\item [Adverbs:] Adverbs and adjectives such as "easily", "generally", etc should be removed because they are unspecific e.g. the statement "can be easily implemented" depends very much on the developer. 
	\item [Articles:] "A" and "an" are indefinite articles; they should be used if the subject is unknown. "The" is a definite article; which should be used if a specific subject is referred to. For example, the subject referred to in "allocated by the coordinator" is not determined at the time of writing and so the sentences should be changed to "allocated by a coordinator".
	\item [Avoid brackets:] Brackets should not be used to hide sub-sentences, examples or alternatives. The problem with this use of brackets is that it is not specific and keeps the reader guessing the exact meaning that is intended. For example "... system entities (users, networks and services) through ..." should be replaced by "... system entities such as users, networks, and services through ...".
	\item [Figures:] Figures and graphs should have sufficient resolution; figures with low resolution appear blurred and require the reader to make assumptions.
	\item  [Captions:] Use captions to describe a figure or table to the reader. The reader should not be forced to search through text to find a description of a figure or table. If you do not provide an interpretation of a figure or table, the reader will make up their own interpretation and given Murphy's law, will arrive at the polar opposite of what was intended by the figure or table.
	\item [Backgrounds:] Backgrounds of figures and snapshots of screens should be light. Developers often use terminals or development environments with dark backgrounds. Snapshots of these terminals or developments are difficult to read when placed into a report. 
	\item [Titles:] Titles of section should never be followed immediately by another title e.g. a title of a chapter should be followed by text describing the content and relevance of the sections of the chapter and could then be followed by the title of the first section of the chapter.
	\item [Punctuation:] A statement is concluded with a period; a question with a question mark.  
	\item [Spellcheckers:] Use a spellchecker!
\end{description}


\section{Figures} 

The arranging of figures in Latex can lead to spending a lot of time on minor issues e.g. positioning a figure in a specific location on a page, fixing minor issues with an exact size of a figure, etc. Figure~\ref{fig:ImageOfAChick} provides a simple example that demonstrates the use of one of two macros for handling figures, called {\it includefigure}; the other macro,  {\it includescalefigure}, is demonstrated in chapter~\ref{chap:Evaluation}. Figures should always be readable without magnification when printed and the resolution of an image should be sufficient to provide a clear picture when printed.

\includefigure{fig:ImageOfAChick}{An Image of a chick}{A caption should describe the figure to the reader and explain to the reader the meaning of the figure. A Sub-clause of Murphy's Law: If the interpretation of a figure is left to a reader, the reader will misinterpret the figure, feel insulted or decide to ignore it. Do not leave it up to the reader!}{image.png}


\section{Structure \& Contents}

At the end of the introduction, a layout of the structure and the contents of the following chapters should be provided for the reader. The overall goal of all descriptions of contents that follows these descriptions is to prepare the reader. The reader should not be surprised by any content that is being presented and should always know how content that is currently being read fits within an overall dissertation.