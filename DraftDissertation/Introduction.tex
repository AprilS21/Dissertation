\chapter{Introduction}

\section{Motivation}

The motivation for this dissertation on the analysis of randomness of Google/Apple Exposure Notification (GAEN) keys spreads across multiple areas including cryptography, public health and digital privacy. Its primary motive stems from the fundamental importance of randomness in cryptography as it ensures the unpredictability and security of cryptographic keys and encryption protocols. Non randomness in encryption can result in vulnerabilities to the confidentiality and integrity of sensitive data. By assessing the randomness of the GAEN keys, ? \newline

Exposure notification and contact tracing systems like GAEN have been used across the world as apps in an effort to combat the spread of Covid-19. GAEN uses Bluetooth technology to exchange cryptographic keys between devices in the proximity of the user and these keys allow users to be alerted if they were potentially exposed to the virus. These apps claim to guarantee to provide user privacy and anonymity. The randomness of the keys being used in these systems is essential to ensure confidentiality of the users.\newline

The GAEN apps provide a unique situation where, due to the design of the GAEN system, a substantial volume of keys generated by devices such as smartphones are readily available. Unlike most cryptographic systems where the keys are not published to the public, the keys generated in the GAEN apps are publicly accessible. These keys have been generated on a wide variety of devices across the world and across different manufacturers and operating systems. Notably, these devices use the same pseudo random number generators (PRNGs) to generate the GAEN keys for other cryptographic applications, FOR EXAMPLE. The quality of the randomness produced by PRNGs is integral to the security of the encryption algorithms that use them and a lack of randomness can lead to predictability which can compromise security. Therefore, the GAEN keys present a rare opportunity to not only analyse and evaluate a large dataset of keys for randomness, but also to give insight into the PRNGs being used in all these devices. \newline

In essence, this dissertation is motivated by an opportunity to explore the security and privacy implications of exposure notification systems by testing the randomness of GAEN keys. This paper aims to contribute to enhancing the trustworthiness of these apps and and validating the privacy claims of the system??.  



\section{Explanation of Terms (rephrase)}

\begin{table}[htbp]
\begin{tabular}{|p{0.35\linewidth}|p{0.55\linewidth}|}
\hline
\textbf{Term}               & \textbf{Definition}                                                                                               \\ \hline
Binary Sequence             & A sequence of ones and zeros                                                                                      \\ \hline
Entropy                     & A measure of unpredictable randomness~(\cite{zolfaghari2022odyssey})                                              \\ \hline
Kolmogorov-Smirnov (KS) Test    & A statistical test that may be used to determine if a set of data comes from a particular probability distribution~(\cite{nist}) \\ \hline
Word                        & Define for dieharder tests                                                                                         \\ \hline
Alphabet                    & Define for dieharder tests                                                                                         \\ \hline
Rank (of a matrix)         & Define for dieharder tests                                                                                         \\ \hline
\end{tabular}
\end{table}
