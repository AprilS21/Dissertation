\chapter{Design}

\section{Challenges}
\subsection{Review of Test Suites}
\subsection{Interpretation of Results}

\section{Methodology}
\subsection{Data Preparation}

The data used in this project was sourced from the Testing Apps for COVID-19 Tracing (TACT) project by Farrell and Leith CITE 2. The TACT project was a study on whether the BLE used in GAEN-based contact tracing applications was effective at identifying users who were in proximity for long enough to be deemed as exposed to covid, if one of the users was later positive for the virus. The project ran from April 2020 until September 2023 and a number of reports were written on the findings CITE some examples. \newline

While the project was ongoing, the TEKs being published in 33 regions, including Ireland, Germany and Brazil, were downloaded hourly. This resulted in a huge amount of data and allowed for insight into the functioning of these apps. The TEKs downloaded were in a large number of zip files. \newline

In order to get the keys in the correct format to test, I first extracted keys from zipped files. Following this, duplicate keys were identified and removed, resulting in a file composed  of only the unique keys. This significantly reduced the data size, from an initial 56GB of all the keys in the zip files, down to 4GB, removing a substantial amount of duplicate keys. The final dataset consists of a total of 129 million unique TEKs in ascii-hex format??, allowing for efficient analysis of the keys.

\subsection{Chosen Test Suite}
\subsection{Other Tests}
\subsection{Random Dataset}